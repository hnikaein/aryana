
% -------------------------------------------------------
%  Abstract
% -------------------------------------------------------


\pagestyle{empty}

\شروع{وسط‌چین}
\مهم{چکیده}
\پایان{وسط‌چین}
\بدون‌تورفتگی
متیلاسیون سیتوزین به علت تأثیرگذاری بالا بر روی فرایندها زیستی مختلف از اهمیت شایانی برخوردار است. نمایان شدن متیلاسیون سیتوزین‌ها با افزودن محلول سدیم بایسولفیت به خوانده‌های بدست آمده از ژنوم، صورت می‌گیرد و هم‌ردیفی بادقت و کارای چنین خوانده‌هایی نقش اساسی در محاسبهٔ درصد متیلاسیون دارد.\\
روش‌های مختلفی تا به حال برای این مسئله مطرح شده است که تمرکز هرکدام بریکی از زمینه‌های سرعت، دقت و کارایی بوده است. در این پایان‌نامه، راه‌کار جدیدی برای این مسئله با گسترش آریانا، که ابزاری برای هم‌ردیفی خوانده‌ها است، ارائه خواهد شد که بر افزایش دقت هم‌ردیفی خوانده‌های بایسولفیت و استفاده از خواص زیستی از جمله خواص سیتوزین‌ها در جزایر و نواحی $CpG$، تأکید می‌کند.\\
با استفاده از این راه‌کار دقت هم‌ردیفی و توانایی هم‌ردیف‌سازی خوانده‌های بایسولفیت، به خصوص در مواردی که خوانده‌ها از نواحی ارزشمند ژنوم که همان \لر{CpG context}ها هستند افزایش یافته است و درصد متیلاسیون سیتوزین‌های خوانده‌های شبیه‌سازی‌شده با اختلاف قابل قبول با مقادیر حقیقی بدست می‌آید.\\
\پرش‌بلند
\بدون‌تورفتگی \مهم{کلیدواژه‌ها}: 
بایسولفیت، هم‌ردیفی، متیلاسیون، جزایر $CpG$، آریانا 
\صفحه‌جدید
