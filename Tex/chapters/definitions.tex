
\فصل{مفاهیم اولیه}

در این فصل به تعریف مفاهیمی می‌پردازیم که در پایان‌نامه مورد استفاده قرار گرفته‌اند.

%----------------------------- مقدمه ----------------------------------


\قسمت{رشته‌ی $DNA$}

$DNA$ مولکولی است که اطلاعات ژنتیکی مورد نیاز برای رشد و فعالیت همهٔ ارگانیسم‌ها و برخی از ویروس‌ها را کد می‌کند.
 $DNA$ نخستین بار در سال ۱۸۷۰ توسط فردریک میشر\پاورقی{Friedrich Miescher} از هستهٔ سلول استخراج و شناسایی گردید. $DNA$ ساختار دو رشته‌ای دارد و این دو رشته مانند زیپ به هم متصل شده و حول یک محور مشترک پیچیده شده‌اند.\\
$DNA$ یک پلیمر بسیار طویل است که از تکرار واحدهایی به نام نوکلئوتید\پاورقی{Nucleotide} به دست می‌آید. هر نوکلئوتید از یک باز نوکلئوتیدی شاملِ نیتروژن، تشکیل شده است که این بازها آدنین ($A$) , تیامین ($T$) ، سیتوزین ($C$) ویا گوانین ($G$) هستند.\\


نوکلئوتیدها در یک زنجیره توسط پیوندهای کوالانسی بین قند یک نوکلئوتید و فسفات نوکئوتید بعدی به یکدیگر متصل شده‌اند. بر اساس قوانین $base pairing$ پیوند هیدروژنی، دو نوکلئوتید از دو رشته $DNA$ را به هم متصل میکند. این پیوند به صورتی برقرار می‌گردد که ادنین و تیامین و همینطور سیتوزین و گوانین روبروی هم قرار بگیرند.


\شروع{شکل}[t]
\centerimg{DNA}{7cm}
\شرح{ساختار $DNA$}
\برچسب{شکل:ساختار دی‌ان‌ای}
\پایان{شکل}


\قسمت{همانندسازی $DNA$}

برای اینکه وراثت امکان‌پذیر باشد، ژن‌ها باید توانایی همانند‌سازی داشته باشند.	ژن ها هر زمان که سلول تقسیم می‌شود باید کپی شوند و هر یک از دو سلول فرزند یک کپی از اطلاعات زیستی والد را دریافت می‌کنند.\\
 در همانند سازی $DNA$، دو رشته آن به کمک آنزیمی مانند زیپ از یکدیگر جدا می‌شوند و سپس از روی هر رشته، رشته‌ٔ جدیدی ساخته می‌شود. کلید ساخت رشته جدید در این است که روبروی هر باز، باز مکمل آن قرار می‌گیرد، به این ترتیب با استفاده از نوکلئوتیدهای آزاد که در سیتوپلاسم وجود دارند‌، در مقابل $A$، باز $T$ و در مقابل $C$ باز $G$ قرار می‌گیرد و در آخر دو کپی کاملا مشابه والد ساخته می‌شود
~\cite{brown2012introduction}.
 

\شروع{شکل}[t]
\centerimg{Replication}{7cm}
\شرح{همانندسازی $DNA$}
\برچسب{شکل:همانندسازی دی‌ان‌ای}
\پایان{شکل}

 
 
 \قسمت{ژنوم}
 
 اطلاعات ژنتیکی حمل شده توسط $DNA$ در توالی و ترتیب خطی $DNA$ به واحد‌های عملکردی مجزایی به نام ژن‌ها تقسیم شده است که به طور شاخص حدود ۵۰۰۰ تا ۱۰۰۰۰۰ نوکلئوتید طول دارند. به مجموعهٔ این ژن‌ها ژنوم گفته می‌شود.

\قسمت{خوانده}

یک مولکول $DNA$ که در سلول‌های زنده موجود است بسیار بزرگ است و ممکن نیست که چنین مولکول بزرگی در تنها یک آزمایش بدست آید. استراتژی فعلی توالی‌یابی رشته $DNA$ این است که مولکول بزرگ آن به قطعات کوچک شکسته شود و هرکدام از آنها جداگانه توالی‌یابی شوند. به این قطعات $fragment$ یا خوانده\پاورقی{read} می‌گویند.

\قسمت{متیلاسیون $DNA$}

متیلاسیون $DNA$ شامل اضافه شدن یک گروه متیل به انتهای کربن سیتوزین است. به سیتوزینی که گروه متیل به آن اضافه شده است سیتوزین متیله گفته می‌شود (شکل~\رجوع{شکل:متیلاسیون دی‌ان‌ای})
~\cite{krueger2012dna}.

\شروع{شکل}[t]
\centerimg{methyl}{7cm}
\شرح{متیلاسیون $DNA$}
\برچسب{شکل:متیلاسیون دی‌ان‌ای}
\پایان{شکل}


\قسمت{خوانده‌های بایسولفیت‌شده}

شامل خوانده‌هایی است که در محلول سدیم بایسولفیت قرار گرفته‌اند که طی این عمل سیتوزین‌های غیر متیله، ابتدا به یوراسیل و سپس به تیامین تبدیل می‌شوند. در این بین مفاهیم دیگری مطرح می‌شوند که به شرح زیر است:

\شروع{شمارش}

\فقره CpG context:
CpG context به معنی سیتوزین‌هایی می‌باشد که بلافاصله بعد از آن  G آمده است. بعد از یک سیتوزین در یک رشته، هم می‌تواند A ,T و یا C بیاید. بنابراین تمام سیتوزین‌ها در CpG contextها نمی‌باشند ولی تمام CpGها به طور مشخص در CpG ها هستند(Gardiner,1987).

\فقره CpG islands:
تعریف کاربردی از CpG island ها به یک ناحیه به همراه حداقل 200 جفت باز برمی گرددکه درصد تعداد CGها بیشتر از ۵۰٪ باشد و نرخ مشاهده شده مورد انتظار CpG بیشتر از ۶۰٪ باشد. منظور از نرخ مشاهده شده مورد انتظار عددی است که از فرمول (2-1) بدست می آید(Gardiner,1987).

\پایان{شمارش}

\قسمت{Sam File}

CpG islands:
تعریف کاربردی از CpG island ها به یک ناحیه به همراه حداقل ۲۰۰ جفت باز برمی گرددکه درصد تعداد CGها بیشتر از ۵۰٪ باشد و نرخ مشاهده شده مورد انتظار CpG بیشتر از ۶۰٪ باشد. منظور از نرخ مشاهده شده مورد انتظار عددی است که از فرمول (2-1) بدست می آید(Gardiner,1987).

\قسمت{رشته‌ی CIGAR}

رشته‌ای است برای مشخص کردن اینکه کدام باز خوانده با کدام باز ژنوم مرجع همردیف شده است . این رشته از حروف i,d,m و اعداد طبیعی تشکیل شده است که ترکیب هر حرف و عدد نشان دهنده به ترتیب تعداد درج‌های متوالی ، تعداد حذف‌های متوالی و تعداد تطابق / عدم تطابق‌های متوالی است و این جفت های عدد و کاراکتر به صورت پشت سر هم ظاهر می‌شوند.

\قسمت{توالی‌یابی نسل بعد}

تکنولوژی (Next Generation DNA Sequencing Technology Technique) NGS مجموعه ای از تکنیک های جدید برای خواندن توالی DNA  می‌باشد که از نظر دقت  افزایش و هزینه، کاهش قابل توجهی نسبت به تکنیک‌ها و متدهای قبلی دارد. در این تحقیق از داده‌های بدست آمده از این تکنولوژی، استفاده می‌شود(Ansorge,2009).

\قسمت{Sequence Alignment}

یک هم‌ردیفی توالی٬ روشی برای پشت هم قرار دادن توالی‌های DNA، RNA یا پروتئین است تا مناطق شباهت که ممکن است علت روابط رفتاری٬ ساختاری و یا تکاملی میان توالی‌ها باشند را شناسایی کند.

\قسمت{DNA Sequencing}

همانطور که گفته شد، DNA یک زنجیره خطی از چهار نوکلوتید می‌باشد. اطلاعات ژنتیکی DNA در توالی این نوکلئوتیدها، رمز شده است. پروسه تعیین توالی نوکلئوتید‌ها در مولکول DNA راDNA sequencing  گویند. متدها و تکنولوژی‌های مختلفی برای یافتن توالی DNA استفاده شده است (Waterman,1995).

\قسمت{آریانا}

آریانا یک برنامهٔ هم‌ردیف‌ساز است که از الگوریتم Burrows Wheeler برای نگاشت خوانده‌ها به ژنوم مرجع استفاده می‌کند. از نقاط برتری این هم‌ردیف‌ساز سرعت و دقت بالا است که در مقایسه با هم‌ردیف‌سازهای موجود قابل توجه است. به خصوص نتایج آزمایشات انجام شده نشان می‌دهد با افزایش طول توالی‌ها سرعت آریانا نسبت به سایر روش‌های مورد بررسی برتری دارد. با توجه به اینکه با پیشرفت روش‌های توالی‌سازی طول توالی‌های ساخته شده روز به روز بلندتر می‌شوند این برتری اهمیت بیشتری پیدا می‌کند.

%----------------------------- مقدمه ----------------------------------



