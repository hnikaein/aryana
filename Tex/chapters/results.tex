
\فصل{راه حل}

\قسمت{الگوریتم}
روشی که برای هم‌ردیفی خوانده‌های بایسولفیت در aligner های امروزی مانند \لر{Bismark} و \لر{Brat} و \لر{BS-Seeker2} رایج است تبدیل تمامی سیتوزین‌های ژنوم مرجع و تمامی سیتوزین‌های خوانده به تیامین و پس از آن هم‌ردیف کردن خوانده‌ها با ژنوم مرجع با استفاده از $Bowtie$ یا روشهای $hashing$ است . مشکلی که در این روش وجود دارد این است که این aligner ها دقت کافی با توجه به اهمیتِ پیدا کردن نقاط متیله شده پیدا ندارند و در این ابزارها برای بالا بردن دقت هم‌ردیفی و پیدا کردن مکان درست متناظر با هر خوانده در ژنوم مرجع، محدودیت وجود دارد. مشکلی که در این روش‌ها وجود دارد در این نکته نهفته است که در هم‌ردیف کردن سیتوزین با تیامین چهار حالت ممکن است رخ بدهد (شکل ~\رجوع{شکل:حالات قابل قبول}):

\شروع{فقرات}
\فقره هم‌ردیفی یک سیتوزین در ژنوم با یک سیتوزین در خوانده 
\فقره هم‌ردیفی یک سیتوزین در ژنوم با یک تیامین در خوانده 
\فقره هم‌ردیفی یک تیامین در ژنوم با یک تیامین در خوانده 
\فقره هم‌ردیفی یک تیامین در ژنوم با یک سیتوزین در خوانده 
\پایان{فقرات}

سه حالت اول برای خوانده‌های بایسولفیت امکان‌پذیر است؛ اگر سیتوزین خوانده متیله باشد حالت اول پیش می‌آید، اگر متیله نباشد حالت دوم و حالت سوم نیز هم‌ردیفی $T$ با $T$ است و مشکلی ندارد. حالت آخر در هیچ شرایطی امکان‌پذیر نیست و این هم‌ردیفی نمی‌تواند قابل قبول باشد ولی باتوجه به اینکه در روش گفته شده تمامی سیتوزین‌ها چه در خوانده و چه در ژنوم به تیامین تبدیل می‌شوند، حالت چهارم از بقیه حالات قابل تمییز دادن نیست و به عنوان یک هم‌ردیفی قابل قبول در این روش‌ها پذیرفته می‌شود.


\شروع{شکل}[t]
\centerimg{3CT}{7cm}
\شرح{حالات قابل قبول نگاشت C به T}
\برچسب{شکل:حالات قابل قبول}
\پایان{شکل}


تغییرات بایسولفیت و تبدیلات نامتقارن سیتوزین‌ها، فضای جست و جو را به صورت قابل توجهی افزایش می‌دهد. رشته های مثبت و منفی که در واکنش با بایسولفیت تغییر یافته‌اند دیگر مکمل یکدیگر نیستند، زیرا تغییرات بایسولفیت فقط در سیتوزین‌های غیرمتیله رخ می‌دهد.
\\
تفکر در این موضوع ما را بر این داشت که روشی جدید ارائه کنیم که در آن ناآگاهانه ژنوم و خوانده‌ها را تغییر ندهیم و عوامل زیستی ناحیه‌ای که یک باز در آن قرار گرفته و مشاهداتی که از خواص این نواحی در مورد متیله شدنشان بدست آمده است، را در ایجاد تغییرات در ژنوم برای تطابق بیشتر با خوانده‌های بایسولفیت اعمال کنیم.
\\
سیتوزین‌های \لر{CpG context} با احتمال ۹۰٪ متیله هستند و سیتوزین‌های خارج این نواحی در حدود ۱ درصد. به علت اهمیت بالای متیلاسیون این نواحی اهمیت بسیاری پیدا می‌کنند. از طرف دیگر جزایر $CpG$ از قاعده فوق مستثنا هستند و درصد متیلاسیون در این نواحی بسیار پایین است
~\cite{meissner2008genome}.
\\
ما در این روش ژنوم‌های مختلفی برای پشتیبانی حالات مختلفی که می‌توان برای سیتوزین‌ها در نظر گرفت، تولید می‌کنیم. به جای آنکه تمامی سیتوزین‌ها را به تیامین تبدیل کنیم به صورت انتخابی و با توجه به مکانی که آن سیتوزین در آن قرار گرفته است این کار را انجام می‌دهیم؛ به این صورت که فرض را بر این می‌گذاریم که سیتوزین‌هایی که در نواحی \لر{CpG context} قرار دارند متیله هستند بنابراین پس از بایسولفیت شدن تغییری نمی‌کنند و سیتوزین‌هایی که در خارج از این نواحی هستند با احتمال خوبی متیله نیستند و پس از بایسولفیت شدن به تیامین تبدیل می‌شوند. همچنین با توجه به اینکه سیتوزین‌های درون جزایر $CpG$ با احتمال بیشتری متیله نیستند، سیتوزین‌هایی که درون نواحی \لر{CpG context} و جزایر $CpG$ هستند را به همراه سیتوزین‌های بیرون  \لر{CpG context} به تیامین تبدیل کردیم و بقیه بدون تغییر باقی ماندند.
\\
ژنومی که با تغییرات فوق بدست می‌آید برای هم‌ردیفی تمامی خوانده‌های بایسولفیت کافی نیست. علت آن این است که در بدست آوردن خوانده‌ها حتی با روش‌های نسل جدید خطا وجود دارد و ممکن است که در خوانده‌ای تیامین به اشتباه سیتوزین گرفته شود و در فرآیند هم‌ردیفی آن خوانده مشکل بوجود آورد. دلیل دیگر این است که در سه حالت فوق ما فرض کردیم که تمامی سیتوزین‌های خوانده‌های بیرون \لر{CpG context} و یا درون جزایر $CpG$ متیله نیستند و تمامی خوانده‌های درون ناحیه و خارج از جزایر متیله‌اند ولی در واقعیت ما این قطعیت را نداریم. برای جبران این تفاوت ما علاوه بر ژنوم فوق ژنوم دیگری تولید می‌کنیم و در آن تمامی سیتوزین‌ها را به تیامین تبدیل می‌نماییم. بدین صورت خوانده‌هایی که با ژنوم قبل نتوانند هم‌ردیف شوند با این ژنوم خواهند شد. برای اینکه هم‌ردیفی با خواص زیستی تطابق بیشتری داشته باشد در انتخاب هم‌ردیفی نهایی به ژنوم قبل اولویت داده می‌شود.
\\
علاوه بر این دو ژنوم نیاز این دیده می‌شود که برای اینکه خوانده‌های بیشتری را بتوانیم هم‌ردیف کنیم، خوانده‌ها با ژنوم اصلی نیزهم‌ردیف شوند. دلیل آن هم این است که ممکن است نواحی‌ای وجود داشته باشند که سیتوزین‌ها علی‌رغم اینکه $CpG$ نیستند کاملا متیله باشند و ظاهر این نواحی مشابه ژنوم اصلی باشد. دلیل بعدی این موضوع این است که عملیات بایسولفیت روی همهٔ خوانده‌ها کامل عمل نمی‌کند و ممکن است یک سیتوزین غیرمتیله به تیامین تبدیل نشود. برای پوشش دادن این حالات نیز بهتر است که از ژنوم اصلی برای هم‌ردیفی استفاده شود.
\\
علاوه بر ژنوم‌های فوق، لازم دیدیم که ژنوم دیگری را نیز در نظر بگیریم که در آن $CpG$های درون جزیره را مانند دیگر $CpG$ها متیله در نظر بگیریم. به این دلیل که متیله نبودن $CpG$ها در جزیره یک رخداد قطعی نیست و این امکان وجود دارد که خوانده‌ای وجود داشته باشد که مطابق انتظار ما نباشد. تلاش ما بر این بود که تا جای ممکن تمامی حالات ممکن برای خوانده‌ها را مدنظر قرار بدهیم. لازم به ذکر است که پس از انجام آزمایش‌های فراوان با خوانده‌های شبیه‌سازی‌شده و واقعی، نتیجهٔ لازم را از این ژنوم اضافه دریافت نکردیم (جدول ~\رجوع{جدول:نگاشت}) و وجود آن را بی‌مورد احساس کرده و آن را از فرآیند حذف نمودیم.


\شروع{لوح}[t]
\تنظیم‌ازوسط

\شروع{جدول}{|c|c|c|c|c|}
\خط‌پر 
\سیاه ژنوم شمارهٔ ۳ & \سیاه ژنوم شمارهٔ ۲ & \سیاه ژنوم شماره ۱ & \سیاه ژنوم اصلی & \سیاه خوانده‌ها \\ 
\خط‌پر \خط‌پر 
۰.۰۱ درصد & ۱.۵ درصد & ۴۳.۶ درصد & ۱۳.۵ درصد & \لر{Mouse Neural Progenitor data} \\ 
\خط‌پر
\پایان{جدول}

\شرح{میزان نگاشت به هر ژنوم در رشتهٔ مثبت}
\برچسب{جدول:نگاشت}
\پایان{لوح}


 این نکته قابل توجه است که در aligner های موجود، مرسوم است که علاوه بر ژنوم، تمامی سیتوزین‌ها در خوانده نیز به تیامین تبدیل شده و سپس هم‌ردیفی انجام می‌شود اما در راه حلی که ما ارائه کرده‌ایم چنین عملی بی‌مورد به نظر می‌رسد. باید توجه داشت که اکثر خوانده‌هایی که هم‌ردیفی ناموفقی با ژنوم ساخته شده در قسمت قبل داشته‌اند، خوانده‌هایی هستند که سیتوزین‌های آن‌ها متیله نبوده و به تیامین تبدیل شده‌اند ولی ما در ژنوم ساخته‌شده به اشتباه آن‌ها را سیتوزین نگه داشته‌ایم. در این حالت دیگر نیازی به تغییر در خوانده نیست و باید بدون تغییر به ژنومی که تمامی سیتوزین‌های آن به تیامین تبدیل شده است، هم‌ردیف گردد.
\\
یکی از موارد بسیار مهم در زمینهٔ هم‌ردیفی خوانده‌های بایسولفیت‌شده، توان‌مندی aligner در هم‌ردیفی خوانده‌های $PCR$ شده است. می‌توان نشان داد که در این حالت، ژنوم‌های مطرح شده کافی نیستند و خوانده‌های مورد نظر را پوشش نمی‌دهند. همانطور که در شکل ~\رجوع{شکل:تبدیل پی‌سی‌آر} پیداست، در این نوع، خوانده‌ها هم از رشتهٔ مثبت و هم از رشتهٔ منفی جمع‌آوری می‌گردند و در دستگاه $PCR$ تکثیر می‌شوند. سیتوزین‌های غیر متیله در این خوانده‌ها، پس از بایسولفیته شدن به تیامین تبدیل می‌شوند. به همین دلیل، اگر خوانده‌ای از رشتهٔ منفی برداشته شده باشد، پس از بایسولفیته شدن معادل ان خواهد بود که در معکوس این خوانده گوانین‌های غیرمتیله به آدنین تبدیل شده باشند. بدیهی‌ست که این مورد را نمی‌توان با ژنوم‌های پیشتر مطرح‌شده پوشش داد. به همین دلیل ژنوم‌هایی ساخته می‌شود که در آنها در رشتهٔ مثبت، گوانین‌ها به آدنین تبدیل می‌گردد. البته ما همچنان خواص زیستی را در این مورد نیز مورد توجه قرار داده‌ایم و برای ساخت این ژنوم مناطق $CG$ و مناطق جزیرهٔ $CpG$ را مد نظر قرار می‌دهیم.


\شروع{شکل}[t]
\centerimg{pcr}{7cm}
\شرح{تبدیل $PCR$}
\برچسب{شکل:تبدیل پی‌سی‌آر}
\پایان{شکل}



\قسمت{انتخاب بهترین هم‌ردیفی}

به منظور انتخاب بهترین هم‌ردیفی برای هر خوانده از میان هم‌ردیفی‌هایی مختلفی که می‌تواند با ژنوم‌های تولید شده داشته باشد، به محاسبه جریمه می‌پردازیم. به این صورت که باز به باز خوانده و ژنوم مرجع را در نظر گرفته و بر اساس جایگاه باز و نوع عدم تطابق، یک جریمه از میان سه جریمه کم، متوسط و زیاد که مقادیر آنها ورودی برنامه هستند در نظر می‌گیریم. جریمه یک خوانده در حال حاضر جمع جریمه‌های بازهای آن است ولی برنامهٔ نوشته شده به این صورت است که قابلیت تغییر تابع محاسبه جریمه به سادگی وجود دارد و می‌توان مدل آماری مناسب برای این کار را بدست آورد و با کمترین تغییرات ممکن آن را به برنامه افزود.

\قسمت{شبیه‌ساز}

یکی از مراحل حساس و دشوار در تولید یک aligner، مرحله تست و آزمایش و ارزیابی نتایج است. برای این منظور امکان استفاده از خوانده‌های واقعی وجود ندارد به این علت که یک مکان قطعی برای هم‌ردیفی خوانده‌ها بر روی ژنوم وجود ندارد و نمی‌توان نظر قطعی در مورد مکان درست خوانده‌ها بر روی ژنوم داد. به همین منظور ابزارهایی برای شبیه‌سازی و تولید خواندهٔ مصنوعی از روی ژنوم وجود دارد. ما در ابتدا از این ابزارها برای تولید خوانده‌های مصنوعی استفاده کرده و آزمایش‌های اولیه را انجام دادیم، اما پس از پیش‌روی بیشتر، نیاز دیدیم که یک شبیه‌ساز تولید کنیم که با دریافت درصدهای متیلاسیون، خطا و $SNP$، خوانده‌های مصنوعی تولید کند و همچنین درصد متیلاسیون هر سیتوزین و گوانین را در یک فایل خروجی دهد. مشکلاتی که در دیگر شبیه‌سازها وجود داشت عدم توجه به جزایر $CpG$ و عدم توانایی تولید خوانده‌ به طور خاص از بعضی نواحی ژنوم بود. امروزه بخش عمده‌ای از پژوهش‌های متیلاسیون $DNA$ برای کاهش هزینه، به صورت $RRBS$ انجام می‌شود. به این صورت که درصد بسیاری زیادی از خوانده‌ها از نواحی جزایر $CpG$ که بخش کوچکی از ژنوم را تشکیل می‌دهند بوسیله افزودن آنزیم‌هایی به $DNA$ آن را بخش‌بندی می‌کنند و سپس خوانده‌های با تراکم بالا از این قسمت‌ها تولید و تست می‌شوند و شبیه‌سازهای موجود معمولا امکان شبیه‌سازی چنین خوانده‌هایی را فراهم نمی‌کنند
~\cite{guo2013bs}.
\\
پس از پیاده‌سازی شبیه‌ساز، با استفاده از آن انواع مختلف خوانده را تولید کرده و در نهایت با هم‌ردیف کردن آنها و بدست آوردن درصدهای متیلیشن، نتایج را با خروجی شبیه‌ساز، مورد مقایسه قرار دادیم.

\قسمت{محاسبه درصد متیلاسیون}

برای بدست آوردن درصد متیلاسیون هر سیتوزین در فایل $Sam$ نهایی خروجی برنامه، که البته برای هر گونه فایل خروجی که در آن فرمت $Sam$ رعایت شده باشد قابل استفاده است، برنامه دیگری نوشته شد.
\\
در این برنامه فایل خط به خط خوانده می‌شود و در ابتدا اگر خوانده‌های $PCR$ هم در فایل موجود باشند، بر اساس $flag$ مشخص‌کنندهٔ رشته‌ای از ژنوم که با آن هم‌ردیف شده و تعداد تبدیل‌های آدنین به گوانین و سیتوزین به تیامین مشخص می‌شود که خوانده مربوط به کدام حالت زیر بوده است:

\شروع{فقرات}
\فقره تعداد تبدیلهای $A$ به $G$ بیشتر از $C$ به $T$ و $flag$ همردیفی ۱۶
\فقره تعداد تبدیلهای $A$ به $G$ بیشتر از $C$ به $T$ و $flag$ همردیفی ۰
\فقره تعداد تبدیلهای $A$ به $G$ کمتر از $C$ به $T$ و $flag$ همردیفی ۱۶
\فقره تعداد تبدیلهای $A$ به $G$ کمتر از $C$ به $T$ و $flag$ همردیفی ۰
\فقره تعداد تبدیلهای $A$ به $G$ حدودا برابر با  $C$ به $T$ و $flag$ همردیفی ۰ یا ۱۶
\پایان{فقرات}

$flag$ برابر با ۱۶ نشان‌دهنده این است که خوانده با رشتهٔ منفی هم‌ردیف شده است و $flag$ صفر نشان‌دهندهٔ هم‌ردیفی با رشتهٔ مثبت است. حالت اول و سوم مربوط به خوانده‌های $PCR$ هستند و حالت دوم و چهارم مربوط به خوانده‌های غیر $PCR$. پس با این شمارش درصورتی که خوانده‌های $PCR$ هم داشته باشیم، می‌توانیم آنها را تمییز دهیم. مشکلی که باقی می‌ماند حالت پنجم است که در آن ما با یک خواندهٔ مبهم روبرو هستیم که در این حالت هم می‌توان از آنها صرف نظر کرد و هم به صورت تصادفی یکی از دو حالت ممکن در نظر گرفت.
\\
پس از مشخص شدن نوع خوانده، سیتوزین‌های ژنوم مرجع با مقادیر متناظر در خوانده‌ها مقایسه می‌شوند و اگر در خوانده $C$ دیده شود به تعداد خوانده هایی که در آنها $C$ متیله بوده است و آن جایگاه در ژنوم را پوشش می‌دهند یک واحد اضافه می‌شود و اگر $T$ دیده شود یک واحد به تعداد $C$های غیر متیله اضافه می‌شود. در واقع پس از پایان این فرآیند برای تمامی خطوط فایل $Sam$ اطلاعاتی که ما به ازای هر سیتوزین در ژنوم مرجع خواهیم داشت عبارتند از:

\شروع{فقرات}
\فقره جایگاه سیتوزین نسبت به ابتدای کروموزوم
\فقره رشته‌ای که خوانده با آن هم‌ردیف شده است 
\فقره تعداد خوانده‌هایی که این سیتوزین را پوشش داده و در آنها این $C$ ،$C$ بوده است. (تعداد خوانده‌های متیله)
\فقره تعداد خوانده‌هایی که این سیتوزین را پوشش داده و در آنها این $C$، $T$بوده است.(تعداد خوانده‌های غیرمتیله)
\پایان{فقرات}
و با داشتن این مقادیر، درصد متیلاسیون هر سیتوزین به صورت جداگانه برای هر رشته بدست می‌آید.

\فصل{پیاده‌سازی}

در اولین گام پیاده‌سازی لازم است که ژنوم‌های مورد نیاز (که در فصل راه‌حل مطرح گردید) از روی ژنوم اصلی ساخته شوند. از آنجایی که معمولا اندازهٔ ژنوم‌ها بسیار بزرگ است، این مرحله زمان زیادی می‌برد. لازم به ذکر است که این گام یک پیش‌پردازش است و برای هر ژنوم این مرحله تنها یک بار انجام می‌شود و پس از آن برای اجرای انواع هم‌ردیفی از همین ژنوم‌ها استفاده می‌شود. این بخش از برنامه به زبان \کد{C++} نوشته شده است.
\\
در گام بعدی، پارامترهایی به برنامهٔ آریانا اضافه گردید تا میان اجرای هم‌ردیفی در حالت بایسولفیت و غیر بایسولفیت تفاوت قائل شود. در حالتی که اجرای بایسولفیت مد نظر باشد، آریانا به ازای هر ژنومِ ساخته شده یک مرتبه هم‌ردیفی انجام می‌دهد. نکتهٔ قابل توجه آن است که برنامهٔ آریانا به گونه‌ای تغییر داده شده است که پیش‌پردازش‌ها و پس‌پردازش‌های مشترک میان هم‌ردیفی‌ها تنها یک بار صورت گیرد که هزینهٔ کمتری پرداخت شود. خروجی این مرحله ۵ فایل $Sam$ است که حاصل ۵ بار هم‌ردیفی است.
\\
در گام آخر باید از میان هم‌ردیفی‌های صورت گرفته (۵ فایل) بهترین را انتخاب کنیم. بدیهی‌ترین روش برای انجام این مورد آن است که به ازای هر خوانده، تمامی فایل‌ها را بررسی کرده و بهترین هم‌ردیفی را انتخاب کنیم. واضح است که انجام این عمل به خودی خود امکان‌پذیر نیست، چرا که نمی‌توان ۵ فایل با ابعاد بسیار بزرگ را در حافظه نگه داشت و همچنین نمی‌توان به ازای هر خوانده تمامی فایل‌ها را یک بار جستجو کرد. به همین منظور در ابتدا فایل‌ها را بر اساس نام خوانده‌ها، توسط دستور \کد{sort} سیستم‌عامل لینوکس\پاورقی{Linux} مرتب می‌کنیم. سپس هر سطر از هر ۵ فایل را می‌خوانیم و از میان آن‌ها بهترین هم‌ردیفی را انتخاب کرده و نتیجه را مستقیما در خروجی چاپ می‌کنیم. در این حالت در هر لحظه تنها ۵ خوانده در حافظه وجود دارند و مشکل حافظه برطرف می‌گردد. قابل توجه است که آریانا، علی‌رغم دیگر alignerها، به ازای خوانده‌های ناموفق در هم‌ردیفی نیز سطری در فایل نهایی قرار می‌دهد. به همین دلیل، ترتیب خوانده‌ها پس از مرتب‌سازی در تمامی ۵ فایل یکسان است.

\قسمت{تابع جریمه}

تشخیص نواحی \لر{CpG context} بر این اساس است که آیا بعد از هر سیتوزین در رشتهٔ مثبت گوانین آمده است یا خیر و در صورت مشاهدهٔ $CG$ یک ناحیه در نظر گرفته می‌شود. مکان‌های جزایر $CpG$ را در حافظه به صورت مرتب‌شده ذخیره کرده و برای بررسی کردن اینکه یک سیتوزین در جزیره $CpG$ قرار دارد یا خیر بر روی داده ساختاری که مکان شروع و پایان جزایر را نگه داشته‌، جستجوی دودویی صورت می‌پذیرد.
\\
در تابع فعلی برای محاسبه جریمهٔ بازها، اگر باز در خوانده و ژنوم هر دو $C$ و بعد از آن $G$ آمده باشد و در جزیره $CpG$ قرار گرفته باشد، جریمه متوسط (زیرا معمولا سیتوزین‌های درون جزایر غیر متیله‌اند) و در خارج جزیره جریمه کم اختصاص می‌یابد (به این دلیل که در این حالت انتظار می‌رود خوانده متیله باشد پس سیتوزین، بعد از بایسولفیت سیتوزین باقی می‌ماند) و در خارج \لر{CpG context} جریمه زیاد اختصاص می‌یابد (به این دلیل که در این حالت انتظار می‌رود خوانده متیله نباشد).
\\
در صورتی که باز در خوانده $T$ و مقدار متناظر در ژنوم $C$ باشد در حالت اول جریمه صفر و در حالت دوم جریمه کم اختصاص داده می‌شود(به این دلیل که در این حالت انتظار می‌رود خوانده متیله باشد پس سیتوزین، بعد از بایسولفیت باید سیتوزین باقی بماند) و در خارج \لر{CpG context} جریمه کم اختصاص می‌یابد(به این دلیل که در این حالت انتظار می‌رود خوانده متیله نباشد).
\\
برای حالات درج و حذف نیز بیشترین جریمه ممکن در نظر گرفته می‌شود چون درج و حذف به وضوح حالت نامطلوبی برای ما به حساب می‌آید و لازم است که بین این حالات و حالت تطابق تفاوت محسوسی قائل شویم.
\\
به طور کلی خوانده‌ها به صورت ترتیبی از فایل خوانده می‌شوند و بر اساس رشته سیگار آنها مشخص می‌شود که کدام بازهای آنها با حذف و اضافه و کدام بازها با تطبیق یا عدم تطبیق هم‌ردیف شده‌اند و سپس به روش توضیح داده شده جریمه‌ کلی یک خوانده محاسبه می‌شود.

\قسمت{محاسبه درصد متیلاسیون}

به دلیل ابعاد بزرگ مساله و محدودیت حافظه‌ای که وجود دارد خوانده‌های هم‌ردیف شده خط به خط از فایل ورودی گرفته می‌شوند و جایگاه سیتوزین‌ها به همراه مقادیر توضیح داده شده در یک بافر\پاورقی{Buffer} حلقوی به سایز طول بزرگترین خوانده، نگه داشته می‌شوند و در صورت رسیدن به انتهای یک خوانده ، بافر به اندازه اختلاف جایگاه شروع خواندهٔ بعدی با شروع خوانده فعلی خالی می‌شود. برای نگاشت سیتوزین‌های ژنوم به بافر از یک تابع $hash$ خطی ساده استفاده می‌شود.