

\فصل{مقدمه}

متیلاسیون\پاورقی{Methylation} سیتوزین\پاورقی{Cytosine} از بسیاری از جهات از جمله رشد جنینی، رونویسی\پاورقی{Transcription}و ساختار کروماتین\پاورقی{Chromatin Structure} بر بیولوژی انسان تأثیرگذار است . این مورد در گیاهان نیز به همان اندازه، در مواردی چون رونویسی، ترمیم $DNA$\پاورقی{DNA repair} و تفاوت سلولی\پاورقی{Cell Differention}، اهمیت دارد. نکته قابل توجه آن است که متیلاسیون $DNA$\پاورقی{Deoxyribonucleic Acid} در انعطاف و حافظه سیستم عصبی مؤثر است و همچنین متیلاسیون غیر طبیعی $DNA$ عامل بسیاری از بیماری‌ها از جمله آلزایمر و سرطان است. روش‌های مختلف درمانی سرطان در حال توسعه هستند که با هدف اصلاح الگوی متیلاسیون عمل می‌کنند.\\
روش استاندارد اندازه‌گیری متیلاسیون $DNA$، افزودن محلول سدیم بایسولفیت\پاورقی{Sodium Bisulfite} به نمونهٔ برداشت‌شده است که سیتوزین‌های غیرمتیله\پاورقی{Unmethylated} را به یوراسیل\پاورقی{Uracil} (که پس از تکثیر به تیامین\پاورقی{Thymine} تبدیل می‌شود) تبدیل می‌کند. پس از آن $DNA$، توالی‌یابی\پاورقی{Sequencing} می‌گردد و با ژنوم مرجع\پاورقی{Reference Genome} مقایسه می‌شود به طوری که نگاشت $C$ به $C$ نشان‌دهندهٔ متیله بودن و نگاشت $T$ به $C$ نشان‌دهندهٔ غیر متیله بودن است
~\cite{frith2012mostly}.\\
روش‌ها و الگوریتم‌های گوناگونی برای توالی‌یابی خوانده‌های بایسولفیت‌شده ارائه شده‌اند و بر اساس آنها ابزارهایی توسعه یافته‌اند. با این حال ضعف این ابزارها در دقت پایین و همچنین عدم درنظرگیری خواص زیستی است. به همین دلیل ما تلاش نمودیم که با توسعهٔ ابزار آریانا، که یک توالی‌یاب قدرتمند است، و در نظرگیری خواص زیستی موثر بر متیلاسیون دقت را افزایش دهیم.



\قسمت{تعریف مسئله}

در این پروژه ما سعی داشتیم که یک aligner برای خوانده‌های بایسولفیت‌شده بنویسیم . هدف مسئله، نگاشت یک تعداد از رشته‌های متشکل از حروف $A$، $C$، $G$، $T$ بر روی یک ژنوم خاص است که تمامی این رشته‌ها از یک ژنوم بدست آمده‌اند ولی کاملا مشابه آن نیستند و تغییراتی در آنها صورت پذیرفته است. هدف یافتن جایگاه این رشته‌ها در ژنوم با دقت حداکثر و درنظر داشتن محدودیت‌های زمانی و حافظه‌ای با توجه به ابعاد مساله است.\\
در این پروژه سعی ما بر آن بود که با گسترش ابزار آریانا ، قابلیت هم‌ردیفی خوانده‌های بایسولفیت‌شده با ژنوم مرجع را به آن بیفزاییم . همانطور که توضیح داده شد در داده‌های بایسولفیت، سیتوزین‌هایی که متیله نباشند به تیامین تبدیل می‌شوند و بنابراین با ژنوم مرجع متفاوتند. به دلیل اهمیت نقش متیلاسیون و نبود یک راه حل دقیق، تلاش‌های بسیاری برای حل این مسئله شده است
~\cite{krueger2012dna}. 
سعی ما در این پروژه بر آن بود که راه‌حلی که ارائه می‌کنیم بر خلاف سایر aligner ها خواص زیستی موثر در متیلاسیون را در نظر بگیرد و با استفاده از این ویژگی بتواند دقت هم‌ردیفی را افزایش دهد.


\قسمت{اهداف تحقیق}

در این پایان‌نامه سعی می‌شود که مسئله‌ی توالی‌یابی خوانده‌های بایسولفیت‌شده، به کمک ابزار آریانا و با توجه به خواص زیستی اثبات‌شده مورد بررسی قرار گیرد و راه حلی کارا برای آن ارائه شود.

\قسمت{ساختار پایان‌نامه}

این پایان‌نامه شامل پنج فصل است. 
فصل دوم دربرگیرنده‌ی تعاریف اولیه‌ی مرتبط با پایان‌نامه است. 
در فصل سوم مسئله‌ی دورهای ناهمگن و کارهای مرتبطی که در این زمینه انجام شده به تفصیل بیان می‌گردد. 
در فصل چهارم نتایج جدیدی که در این پایان‌نامه به دست آمده ارائه می‌گردد. در این فصل، مسئله‌ی درخت‌های ناهمگن در چهار شکل مختلف مورد بررسی قرار می‌گیرد. سپس نگاهی کوتاه به مسئله‌ی مسیرهای ناهمگن خواهیم داشت. در انتها با تغییر تابع هدف، به حل مسئله‌ی کمینه کردن حداکثر اندازه‌ی درخت‌ها می‌پردازیم.
فصل پنجم به نتیجه‌گیری و پیش‌نهادهایی برای کارهای آتی خواهد پرداخت.
