
\فصل{کارهای پیشین}

روش‌های بسیاری برای هم‌ردیفی خوانده‌های بایسولفیت٬ مانند روش‌های «wild card» و «سه حرفی» معرفی شده‌اند. دو نوع پیاده‌سازی برای روش «wild card» وجود دارد:

\شروع{شمارش}
\فقره در این روش به تمامی سیتوزین‌ها و تیامین‌های خوانده این اجازه داده می‌شود که به سیتوزین ژنوم مرجع نگاشت شوند.
\فقره در روش دوم تمامی ترکیبات سیتوزین و تیامین را برای هر طول seed می‌شمارد و سپس با روش‌های hashing آن را نگاشت می‌کند.

\پایان{شمارش}
در روش «سه حرفی» تمامی سیتوزین‌ها در خوانده‌ها و در ژنوم مرجع به تیامین تبدیل می‌شوند. در هر دوی این روش‌ها می‌توان نگاشت gapped و یا ungapped را بسته به برنامهٔ مورد استفاده٬ پیاده‌سازی کرد.

\قسمت{الگوریتم های ابتدایی}

برنامهٔ  CokusAlignment  از روشی برای نگاست خوانده‌ها به ژنوم Arabidopsis که بر اساس الگوریتم‌های جستجوی درخت بنا شده است که هم از نظر محاسبات و هم از نظر حافظه بسیار ضعیف است. CokusAlignment با سرعت متوسط 25 reads/sec/CPU با ژنوم تقریبا کوچک اجرا می‌شود. لازم به ذکر است که می‌توان با بهینه‌سازی برای پروژه‌های مختلف این سرعت را بهبود بخشید اما در بسیاری از پروژه‌ها چنین کاری امکان‌پذیر نیست. همچنین این روش برای پیاده‌سازی نگاشت‌های gapped و pair-end نیز مناسب نیست. از نظر عملی٬ به دلیل کمبود سرعت و عملکرد٬ این روش قابل استفاده نیست.

\قسمت{ابزار بیسمارک}

هدف ابزار Bismark، یافتن یک تطابق منحصر به فرد با چهار بار اجرای پردازش‌ها به صورت هم‌زمان می‌باشد. در ابتدا readهای بایسولفیت با تغییراتی از نوع C  به T (سیتوزین به تیامین)و از نوع G به  A(گوانین به آدنین) تبدیل شده است (برابر با تغییرات  سیتوزین به تیامین در رشته معکوس ). سپس هر کدام از آنها به صورت معادل از نوع‌های پیش تغییر یافته از ژنوم رفرنس نگاشت می‌شوند که این عمل با استفاده از نرم افزار نگاشت Bowtie و به صورت چهار نمونه موازی صورت می گیرد. این نگاشت‌ها، Bismark را قادر می‌سازد تا به صورت مشخص، رشته اصلی بایسولفیت را مشحص نماید( (Kruger,2011
خروجی نگاشت ابتدایی Bismark شامل یک خط برای هر read و یک تعداد از اطلاعات مفید مانند جایگاه نگاشت، رشته رفرنس که به آن نگاشت شده و read بایسولفیت شده نگاشت می‌باشد. این اطلاعات را می‌توان به عنوان پس‌پردازش درنظر گرفت. همچنین Bismark خروجی متیلاسیون را در بین ناحیه های مختلف CPG,CHG,CHH  در نظر می‌گیرد و این ناحیه‌ها را از هم جدا در نظر می گیرد

\قسمت{ابزار BSmap}

BSmap از این قضیه اصلی که تمام جایگاه‌های سیتوزین‌ که تغییرات نا متقارن C/T (سیتوزین به تیامین ) در آنها رخ می‌دهند ، شناخته شده هستند، برای راهنمایی در هم‌ردیفی readهای بایسولفیت شده استفاده می کند. BSMAP، تیامین‌های موجود درخوانده‌های بایسولفیت شده را به عنوان سیتوزین‌ mask می نماید.(برعکس تغییرات بایسولفیت)که این تغییر را فقط در جایگاه های سیتوزین‌ ژنوم اصلی انجام می دهد درحالیکه کل تیامین‌های دیگر را در خوانده‌های بایسولفیت شده بدون تغییر نگه می دارد.بعد از انجام این تغییرات ،خوانده‌های بایسولفیت شده را به طور مستقیم بر روی ژنوم نگاشت می نماید (Chen,2010).
علاوه بر موارد ذکر شده،BSMAP براساس الگوریتم کارا‌ترHASH table seeding  کار می‌کند به این صورت که ژنوم مرجع را برای تمام k-merهای ممکن شاخص گذاری می نماید که هر کدام از آنها seed خوانده می‌شود. برای یافتن یک نگاشت، تنها جایگاه‌هایی (seed)به طورکامل با بخشی ازخوانده منطبق شده اند جست و جو می‌شوند. با جست و جو در جدول seedها، اکثریت جایگاه‌های غیرنگاشت شده دورریخته می‌شوند و کارایی جست و جو به صورت قابل ملاحظه ای افزایش می یابد(Chen,2010).

\قسمت{کاستی‌های روش‌های پیشین}

این روش خواده‌های بایسولفیت را تغییر نمی‌دهد بلکه mismatch سیتوزین به تیامین را مجاز می‌داند. بزرگ‌ترین کاستی این روش آن است که تعداد نگاشت‌های سیتوزین/تیامین که در یک خوانده می‌تواند شناسایی شود محدود به تعداد mismatchهای مجاز در نرم‌افزار مورد استفاده است. این تعداد می‌تواند توسط mismatchهای واقعی (SNPها) بیشتر کاسته شود و نتیجه را بیش از قبل محدود کند.


