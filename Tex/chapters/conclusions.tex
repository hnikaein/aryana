
\فصل{نتایج}

در این فصل به بررسی عملکرد ابزار تهیه‌شده می‌پردازیم و این ابزار را از جهات مختلف با دیگر ابزارهای موجود مورد قیاس قرار می‌دهیم.

\قسمت{مقایسهٔ زمان اجرا}

در نمودار ~\رجوع{شکل:زمان} زمان اجرای هر سه ابزار \لر{BSmap}، \لر{Bismark} و \لر{BS-Seeker2} برای دادهٔ $benchmark$\پاورقی{http://www.cbrc.jp/dnemulator/bab/} که شامل یک میلیون خوانده به طول ۸۷ و بایسولفیت شده از ژنوم \لر{hg19} است، اندازه گرفته و با زمان اجرای آریانا مقایسه می‌شود که همانطور که از نمودار مشخص است، زمان اجرای آریانا از \لر{Bismark} و \لر{BS-Seeker2} بیشتر بوده است و آن هم به دلیل این است که آریانا خوانده‌ها را با تعداد ژنوم بیشتری از این دو ابزار هم‌ردیف می‌کند(۵ ژنوم).
\\
زمان اجرای \لر{BSmap} نیز به علت ساخت \لر{k-mer}ها از ژنوم بسیار بالا بوده است.
\\
زمان اجرای هم‌ردیفی، به وسیلهٔ دستور \کد{time} اندازه‌گیری و همهٔ aligner ها با ۸ ریسه و روی محیط یکسان با مشخصات ذکر شده اجرا شده‌اند.

\شروع{شکل}[t]
\centerimg{time}{10cm}
\شرح{مقایسهٔ زمان اجرا}
\برچسب{شکل:زمان}
\پایان{شکل}

\قسمت{مقایسهٔ توانایی هم‌ردیف‌سازی و دقت}

در نمودار ~\رجوع{شکل:توانایی هم‌ردیف} درصد خوانده‌های هم‌ردیف شده توسط هر aligner نمایش داده شده است. این درصدها برای خوانده‌های $benchmark$ توضیح داده شده، به دست آمده‌اند و همانگونه که مشخص است آریانا بیشترین تعداد خوانده را هم‌ردیف کرده است.

\شروع{شکل}[t]
\centerimg{2}{10cm}
\شرح{مقایسه توانایی هم‌ردیف‌سازی}
\برچسب{شکل:توانایی هم‌ردیف}
\پایان{شکل}

در نمودار ~\رجوع{شکل:دقت هم‌ردیف} درصد خوانده‌های هم‌ردیف شده به جایگاه‌های صحیح توسط هر aligner نمایش داده شده است. این درصدها برای خوانده‌های $benchmark$ توضیح داده شده و با مقایسه جایگاه نوشته شده در فایل‌های خروجی با جایگاه‌های اصلی که به همراه داده وجود داشت، بدست آمده‌اند و برای آریانا، \لر{BSmap}، \لر{Bismark} و \لر{BS-Seeker2} برابر با ۸۲.۵، ۵۵.۲، ۸۱.۲ و ۷۲.۲ بوده است که آریانا و بعد از آن \لر{Bismark} بیشترین دقت هم‌ردیفی را داشته‌اند.


\شروع{شکل}[t]
\centerimg{1}{10cm}
\شرح{مقایسه دقت هم‌ردیف‌سازی}
\برچسب{شکل:دقت هم‌ردیف}
\پایان{شکل}


در نمودار ~\رجوع{شکل:دقت و توان هم‌ردیف} نیز درصدهای کل خوانده‌های هم‌ردیف شده و خوانده‌های صحیح هم‌ردیف شده مشاهده می‌شود.


\شروع{شکل}[t]
\centerimg{3}{10cm}
\شرح{مقایسه دقت و توانایی هم‌ردیف‌سازی}
\برچسب{شکل:دقت و توان هم‌ردیف}
\پایان{شکل}


\قسمت{مقایسهٔ توانایی هم ردیف‌سازی خوانده‌های $PCR$}

داده‌های نمودار ~\رجوع{شکل:دقت پی‌سی‌آر}، خوانده‌هایی هستند که توسط شبیه‌ساز ما، تولید شده‌اند. خوانده‌های این داده $PCR$ شده و تعداد آنها ۱ میلیون و فاقد خطا هستند. در این نمودار درصد خوانده‌های هم‌ردیف شده و دقت هم‌ردیفی مشاهده می‌شود. آریانا با اختلاف زیادی در دقت هم‌ردیفی پیشرو است (٪۹۵.۸) و  در جایگاه بعدی  \لر{BSmap} با ۴۸٪ دقت قرار دارد.


\شروع{شکل}[t]
\centerimg{4}{10cm}
\شرح{مقایسه دقت در خوانده‌های $PCR$}
\برچسب{شکل:دقت پی‌سی‌آر}
\پایان{شکل}


\قسمت{بررسی دقت محاسبهٔ درصد متیلاسیون}

داده‌های نمودار ~\رجوع{شکل:درصد متیلاسیون} ، خوانده‌های شبیه‌سازی شده از کروموزوم ۱۰ انسان هستند. تعداد این خوانده‌ها ۶۰ میلیون است تا پوشش ژنوم انتخاب شده بالا باشد، همچنین ۳۰٪ از این خوانده‌ها از جزایر $CpG$ داده شده‌اند که متیلاسیون آنها اهمیت بالایی دارند. در این نمودار محور $X$ معادل درصد متیلاسیون خروجی از شبیه‌سازی و محور $Y$ معادل درصدهای محاسبه شده توسط آریانا است. همانطور که انتظار می‌رفت ۳ دستهٔ کلی برای درصد متیلاسیون سیتوزین‌ها وجود دارد که شامل سیتوزین‌های خارج \لر{CpG context} با ۱ درصد متیلاسیون٬ سیتوزین‌های داخل \لر{CpG context} و بیرون از جزیرهٔ $CpG$ با ۹۰ درصد متیلاسیون و سیتوزین‌های درون جزیره با ۳ درصد متیلاسیون هستند. همان‌گونه که انتظار می‌رفت اکثر نقاط بر روی خط با شیب ۱ قرار گرفته‌اند که این نشان می‌دهد که آریانا در اکثر موارد٬ درصد درستی را محاسبه کرده است. تعدادی از نقاط نیز در مکان‌های نادرستی قرار گرفته‌اند که این می‌تواند ناشی از نرخ خطای ۲ درصد و همچنین ۲ میلیون $SNP$ در خوانده‌ها باشد.


\شروع{شکل}[t]
\centerimg{6}{10cm}
\شرح{درصد متیلاسیون واقعی و به دست آمده}
\برچسب{شکل:درصد متیلاسیون}
\پایان{شکل}



داده‌های نمودار ~\رجوع{شکل:خطا متیلاسیون} نیز خوانده‌های شبیه‌سازی شدهٔ قبلی هستند. در این نمودار فراوانی اختلاف‌ درصد متیلاسیون محاسبه شده و درصد متیلاسیون واقعی نشان داده شده است و شکل نمودار و قله با ارتفاع بسیار بلند در صفر بیانگر این است که درصدهای متیلاسیون محاسبه شده بسیار نزدیک به درصدهای واقعی هستند. همانطور که مشهود است با دور شدن از مبدا و افزایش خطا کاهش چشمگیری در فراوانی درصد اختلاف‌ها رخ می‌دهد و نسبت تعداد آنها به کل خوانده‌ها به صفر میل می‌کند.

\شروع{شکل}[t]
\centerimg{5}{10cm}
\شرح{فراوانی اختلاف درصد متیلاسیون}
\برچسب{شکل:خطا متیلاسیون}
\پایان{شکل}


\قسمت{کارهای آینده}

ابزار آریانا این قابلیت را دارد که از ابعاد مختلف گسترش داده شود. تلاش ما بر این است که تا جای ممکن به قابلیت‌ها و امکانات و همچنین دقت هم‌ردیفی خوانده‌های بایسولفیت‌شده بیفزاییم و در ادامه ابعاد دیگری از مسائل مربوط به توالی‌یابی را با ابزار آریانا پشتیبانی کنیم.\\
از دیگر مواردی که تصمیم داریم در آینده به این ابزار بیفزاییم٬ استفاده از مدل‌های آماری پیچیده و کاراتر در محاسبهٔ جریمه و انتخاب بهترین هم‌ردیفی است که می‌تواند تأثیر به‌سزایی در افزایش دقت داشته باشد.